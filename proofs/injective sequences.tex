\documentclass[10pt]{report}
\usepackage[top=3cm, bottom=2.5cm, left=3cm, right=2.5cm] {geometry}
\geometry{letterpaper}                   % ... or a4paper or a5paper or ... 
%\geometry{landscape}                % Activate for for rotated page geometry
%\usepackage[parfill]{parskip}    % Activate to begin paragraphs with an empty line rather than an indent
\usepackage{graphicx}
\usepackage{amssymb}
\usepackage{epstopdf}

	% NOTE: It seems strategic to set the proofs repository as a submodule of any repository making
	% 	use of it.  For articles `some_article' and `other_article' it seems sensible to put the directories
	%	`some_article', `other_article' and `packages' (which contains proofs as a submodule) side by
	%	side.
	% (END OF NOTE)
\usepackage{fancyhdr,lastpage,color}
\usepackage{../packages/bsymb}
\usepackage{../packages/b2latex}
\usepackage{../packages/elogic}

\DeclareGraphicsRule{.tif}{png}{.png}{`convert #1 `dirname #1`/`basename #1 .tif`.png}

\title{Brief Article}
\author{Some properties of sequences}
%\date{}                                           % Activate to display a given date or no date
\pagestyle{fancy}
\begin{document}
\maketitle
%\section{}
%\subsection{}

This is a small note about the proof on a useful property of sequences.  Although very obvious, one such as my self can hardly satisfied by such a statement.  Indeed the \emph{joual} expression "yien qu'a woere on woit ben!" is hardly convincing.  The property in question is that the concatenation of two disjoint injective sequences is also injective.  We will use $s$ and $r$ to denote the two such sequences and express respectively injectiveness and disjointness as follows:

\begin{equation}
	s \textbf{ is injective} \equiv \iter{\forall}{i,j}{}{\neg (\interval{i}{0}{\# s}) \lor \neg (\interval {j}{0}{\# s}) \lor \neg (s.i = s.j) \lor i = j}
\end{equation}

\MACHINE{Name}
\END

\end{document}  